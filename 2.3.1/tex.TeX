\documentclass[a4paper]{article}

\usepackage[english, russian]{babel}
\usepackage[T2A]{fontenc}
\usepackage[utf8]{inputenc}
\usepackage{amsmath, amsfonts, amssymb, amsthm, mathtools}
\usepackage{soul}
\usepackage{multicol}
\usepackage{lipsum}
\usepackage{mwe}
\usepackage[export]{adjustbox}
\usepackage{mathtext}
\usepackage[a4paper, total={18cm, 8in}]{geometry}
\usepackage{wrapfig}


\begin{document}

\title{Лабораторная работа 2.3.1  \par \textbf{Получение и измерение вакуума}}
\author{Журавлёв Максим}
\date{20.02.25}
\maketitle

\vspace{20pt}
Цель работы:\par
1)Измерение объёмов форвакуумной и высоковакуумной частей установки \par
2)Определение скорости откачки системы в стационарном режиме, по ухудшению и улучшению вакуума
\line(6, 0){\textwidth}

\vspace{20pt}
При рассмотрении диффузии, теплопроводности, вязкости газы считают сплошными средами
(основное влияние на движение частиц газа оказывают их столкновения между собой). Однако при рассмотрении
ситуаций, для которых число Кнудсена($Kn = \frac{\lambda}{d}$) порядка 1 и больше, частицы
чаще сталкиваются со стенками сосуда.  \vspace{5pt}

Рассмотрим течение разреженного газа по трубе радиуса r. Пусть молекулы взаимодействуют только со
стенками, тогда примем длину свободного пробега $\lambda = 2r$. Применим основное уравнение диффузии:
\[\frac{dN}{dt} = D\frac{dn}{dx}S\] 
Для стационарного течения $\frac{dN}{dt} = const$, тогда $\frac{dn}{dx} = \frac{n_1 - n_2}{L}$
\[\frac{dN}{dt} = \frac{2}{3}r^3\frac{n_1 - n_2}{L}\sqrt\frac{8RT\pi}{\mu}\]
Получили ф-лу Кнудсена. Учитывая, что dM = mdN, найдём массу газа, протекающего через трубу в единицу времени
\[\frac{dM}{dt} = \frac{4}{3}r^3\frac{n_1 - n_2}{L}\sqrt{2\pi{mkT}} = \frac{4}{3}r^3\frac{P_1 - P_2}{L}\sqrt\frac{2\pi\mu}{RT}\]

При вакуумных измерениях $\frac{dM}{dt}$ считают в единицах PV, $dM = d(PV)\frac{\mu}{RT}$\par
\vspace{3pt}
Для вязкого газа(сплошная среда) расход массы пропорционален $r^4$, а для разреженного $r^3$.

\pagebreak

\textbf{Вакуум}

Различают 3 степени разреженности газа. Низкий вакуум($10^{-2} - 10^{-3}$торр), высокий вакуум($10^{-4} - 10^{-7}$торр), сверхвысокий вакуум($10^{-8} - 10^{-11}$торр).
низкий вакуум переходит в высокий, если $Kn \sim 1$. В работе будут исследоваться как низкий вакуум(с помощью форвакуумного насоса), так и высокий(диффузионный масляный насос) 
\vspace{5pt}

\underline{Установка А}

Откачка воздуха происходит с помощью форвакуумного и диффузионного насосов
\vspace{5pt}

\textit{Форвакуумный насос}

    \begin{center}\includegraphics[width=0.6\textwidth]{лаба2/форвакуум.jpg}
    \end{center}

Форвакуумный насос состоит из ротора, диаметрально разделённого двумя пластинами(А и Б).
При повороте ротора, в зависимости от положения пластин, через один клапан газ поступает в полость между А и Б, 
через другой выпускается. 

Перед работой насос должен сначала откачать собтсвенный объём. После работы насоса важно запустить в него атмосферу

\vspace{5pt}
\textit{Диффузионный насос}

Работа диффузионного насоса построена на увлекании молекул газа парами масла.

\begin{wrapfigure}{l}{0.56\textwidth}
    \centering
    \includegraphics[width=0.35\textwidth]{лаба2/диф.png}

\end{wrapfigure}

\vspace{10pt}
Масло подогревается так что его пары поднимаются и вырываются из сопла. Струя паров
увлекает молекулы газа в трубке до попадания в вертикальную трубу, из которой газ откачивается форвакуумным насосом, а 
масло осаждается на стенках и стекает вниз, обратно в исходный сосуд.

Если зазор между соплом и стенками трубы порядка длины свободного пробега, то молекулы газа через этот зазор увлекаются 
наиболее эффективно, так как все молекулы оказавшиеся в сечении зазора уже не меняют направления свое скорости.

В установке используется диффузионный насос, работающий в 2 ступени(вертикальная и горизонтальная) 

\pagebreak


\textbf{Экспериментальная установка}

Схема состоит из: 1)Форвакуумного насоса(ФН), форвакуумного балона(ФБ), манометра 1(М1), масляный манометр(М) - для получения низкого ваккума.
\par
2)Высоковакуумного насоса(ВН), высоковакуумного балона(ВБ), манометра 2(М2), ионизационного балона(И) - для высокого вакуума

Манометры М1 и М2 образуют термопару. Система регулируется с помощью кранов(К), соединяющих части установки

\textbf{Схема установки}

\begin{center}
    \includegraphics[width=0.8\textwidth]{лаба2/установка.png}
\end{center}

\pagebreak

\textbf{Как происходит откачка}

Определим производительность насоса как $ W = \frac{dV}{dt}$(при данном давлении - объём газа удаляемого из сосуда)
Общее кол-во газа покидающего "откачиваемый объём" определяется помимо мощности насоса: десорбцией с поверхностью, 
проникновением газа через течи и проникновением газа обратно из насоса. Тогда, записывая в единицах PV изменение кол-ва газа:

\[-VdP = (PW - Q_{\text{д}} - Q_{\text{и}} - Q_{\text{н}})dt\]

Считая $Q_{\text{и}}$ постоянной, а $Q_{\text{д}}, Q_{\text{н}}$ независящими от времени, проинтегрируем

\[P - P_{\text{предел}} = (P_0 - P_{\text{предел}}) \exp(-\frac{W}{V}t)\]
считая $P_{\text{предел}}$: $\frac{dP}{dt} = 0$ пренебрежимо малым по сравнению с начальным давлением:

\[P = P_0 \exp(-\frac{W}{V}t)\]

Здесь величина $\frac{V}{W}$ - постоянна, она характеризует $\textit{время}$, за которое удастся откачать систему до около предельного давления

\vspace{8pt}

Скорость откачки системы(W) определяется не только действием самого насоса, но и проводимостью кранов, трубок и иных частей системы

Рассматривая откачивающую установку можно ввести аналогию с электрической цепью, поток газа отождествить с силой тока.

В таком случае пропускная система системы определяется аналогично проводимостям элементов в электрической цепи (для последовательного соединения)

\[\frac{1}{W} = \frac{1}{W_H} + \frac{1}{C_1} + ...\]

здесь за $C_i$ обозначены пропускные способности элементов установки

\textbf{Определение параметров компонентов}

В условиях высокого вакуума характер течения в основном зависит от столкновений
молекул газа со стенками сосуда(длина свободного пробега в таком случае - 2r). В данных условиях для трубы радиуса r

\[\frac{d(PV)}{dt} = \frac{4}{3}r^3\frac{P_1 - P_2}{L}\sqrt\frac{2\pi{RT}}{\mu}\]

Пренебрежём давлением $P_1$ в откачиваемом объёме и определим изменение объёма при давлении $P_2$

\[C_{\text{трубы}} = \frac{dV}{dt} = \frac{4}{3}\frac{r^3}{L}\sqrt{\frac{2\pi{RT}}{\mu}}\]

Так как длина свободного пробега в высоком ваккуме порядка r, то для диффузионного насоса
каждая молекула попавшая в зазор между соплом и стенками трубы увлекается парами.
Тогда скорость откачки насоса можно считать как пропускную способность некоторого отверстия площади зазора.

Для проводимости отверстий нетрудно получить 
\[\frac{dN}{dt} = \frac{1}{4}Snv\]

\[C_{\text{отв}} = \frac{dV}{dt} = \frac{1}{4}Sv\]

\pagebreak

\underline{Измерения} \vspace{2pt}

Измерим объём форвакуумной и высоковакуумной частей установки. Для этого дадим запертому в установке воздуху
при атмосферном давлении расшириться сначалала в форвакуумную часть установки, затем в высоковакуумную. 
Воспользуемся законом Бойля-Мариотта: $PV = const$

$V_0 = 50 cm^3 \hspace{5pt}P_{\text{атм}} = 100.76 \text{КПа} $

Получили:
\[V_{\text{фвм}} = 2,15 \text{л} \hspace{8pt}  V_{\text{ввмъ}} = 1,23 \text{л}\]
\vspace{5pt}

Оценим производительность насоса. Проведём 2 эксперимента по ухудшению и улучшению вакуума

\begin{wrapfigure}{r}{0.6\textwidth}
    \centering
    \includegraphics[width=0.59\textwidth]{лаба2/lnp(t).png}
    \caption{$ln(\frac{P}{P_0});\hspace{5pt} t, c$}
\end{wrapfigure}
\vspace{7pt}

После отключения вакуума от насоса краном 3, вакуум ухудшился до давления $8 * 10^{-4}$ тор.
Затем вв балон снова подключается к насосу.

Ожидаемая экспоненциальная зависимость давления от времени
в процессе откачки изображена на графике lnP(t).

Здесь коэффицент угла наклона определяется производительностью насоса, 
которую выразим из

\[P = P_0 e^{-\frac{W}{V}t}\]

\[k = -0.14, W = 0.17 \frac{\text{л}}{c}\]

Полученную скорость откачки можно использовать в равенстве:

$P_{\text{пр}}W = Q_{\text{д}} + Q_{\text{и}} + Q_{\text{н}}$
\vspace{2pt}


\begin{wrapfigure}{l}{0.6\textwidth}
    \centering
    \includegraphics[width=0.6\textwidth]{лаба2/ухудшение.png}
    \caption{$P, 10^{-5}\text{тор};\hspace{5pt} t, c$}
\end{wrapfigure}
\vspace{7pt}

Для ухудшения вакуума отключим вв балон от насоса с помощью крана 3.

В этом случае 
\[V_{\text{вв}}dP = (Q_{\text{д}} + Q_{\text{и}})dt\]
Определим
$(Q_{\text{д}} + Q_{\text{и}})$ как коэффицент угла наклона p(t), с учётом предыдущего равенства.\par

$k = 0.71 * 10^{-5}$, $(Q_{\text{д}} + Q_{\text{и}})$ = $0.87 * 10^{-5}$ $\frac{PV}{c}$

Предельное давление - $6 * 10^{-5}$ тор
\vspace{5pt}

$Q_{\text{н}} = 0.15 * 10^{-5} \frac{\text{тор*л}}{c}$

\vspace{5pt}
Полученное значение в 6 раз меньше количества воздуха, откачиваемого насосом в секунду при данном давлении,
это значит насос действительно забирает из откачиваемого объёма воздуха больше, чем
через него уходит обратно  


\pagebreak

\textit{Течение воздуха через капиляр}

\[\frac{d(PV)}{dt} = P_{\text{уст}}W - P_{\text{пр}}W\]

$P_{\text{уст}} = 10^{-4}$ тор - установившееся давление после открытия крана К5 

\[\frac{d(PV)}{dt} = 8.5  * 10^{-7} \frac{\text{тор*л}}{c}\]

Сравним со значением, полученным теоретически для течения через трубку в кнудсеновском режиме

\[\frac{d(PV)}{dt} = \frac{4}{3}\frac{r^3}{L}\sqrt{\frac{2\pi{RT}}{\mu}}\]

\vspace{5pt}
$\frac{d(PV)}{dt} = 6.4 * 10^{-8} \frac{\text{тор*л}}{c}$

\vspace{5pt}
Оценим пропускную способность насоса в отдельности от системы. В силу устройства диффузионного насоса
его скорость откачки можно считать как пропускную способность отверстия площади S.
\[C_{\text{отв}} = S\frac{v}{4}\]

\includegraphics[width=0.4\textwidth]{лаба2/зазор}
\includegraphics[width=0.25\textwidth]{лаба2/насос.jpg}

$S \approx 4.4 cm^2$ \hspace{3pt}$\frac{v}{4} \approx 11 \frac{\text{л}}{c*cm^2}$
\vspace{5pt}

$C \equiv W_{\text{насоса}} = 48.7 \frac{\text{л}}{c}$. Можно заметить, что производительность откачивающей системы в данном случае в меньшей степени
зависит от мощности насоса и в большей от проводимости составляющих её элементов(трубы, кран).
\vspace{10pt}

\textbf{Отчёт}

\vspace{5pt}

1) С помощью диффузионного насоса был получен высокий вакуум с предельным давлением $6 * 10^{-5}$тор.
\vspace{5pt}

2) Давление в откачиваемом объёме экспоненциально зависит от времени(нарушается при значениях близких к предельному)
\vspace{5pt}

3) Получено аналитическое значение производительности насоса. Как оказалось, на производительность системы
 в большей степени влияет пропускная способность труб и крана, соединяющих насос и откачиваемый объём

\end{document}