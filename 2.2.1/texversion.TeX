\documentclass{article}

\usepackage[english, russian]{babel}
\usepackage[T2A]{fontenc}
\usepackage[utf8]{inputenc}
\usepackage{amsmath, amsfonts, amssymb, amsthm, mathtools}
\usepackage{soul}
\usepackage{multicol}
\usepackage{lipsum}
\usepackage{mwe}
\usepackage[export]{adjustbox}
\usepackage{array}

\begin{document}
 \title{Лабораторная работа 2.2.1  \par \textbf{Взаимная диффузия газов}}
 \author{Журавлёв Максим}
 \date{06.02.25}
 \maketitle
 
 \pagebreak
 \large
 
 \textit{Дифузия} - самопроизвольное взаимное проникновение веществ друг в друга, вследствие хаотичного теплового движения частиц
 
 \vspace{20pt}
 Рассмотрим двухкомпанентную систему газов(\textit{бинарная смесь})
 \vspace{3pt}

 \huge{Закон Фика}
 \large\[j_a = - D \frac{\partial n_a}{\partial x}\]

 где $j_a$ - плотность потока частиц вещества \textit{a}

 \vspace{8pt}
 
 Равновесное состояние при постоянной концентрации по всему объёму
 
 \vspace{15pt}

 Оценим коэффицент взаимной диффузии $D$
 
 В проводимом опыте $n_{he}$  << $n_{air}$, а также атомы $He$ значительно легче молекул, содержащихся в воздухе($N_2, O_2$), поэтому их
 скорости больше. Тогда диффузию гелия и воздуха можно рассматривать как проникновение молекул гелия в стационарный воздух
 \par
 В таком приближении:
 \[D = \frac{1}{3{n_{air}}\sigma}\sqrt{\frac{8RT}{\pi\mu_{He}}} = \frac{1}{3}\lambda{v_{add}}\]

 $\sigma$ - площадь сечения области столкновения, $\lambda$ - длина свободного пробега частиц гелия, $v_{add}$ средняя тепловая скорость частиц гелия

 \vspace{2pt}

 Так как $n = \frac{P}{kT}$, то $D \propto \frac{1}{P}$ \vspace{3pt}

 Границы применимости:\par
 1) Постоянная температура \par
 2) Длина свободного пробега частиц много меньше характерных размеров установки \par
 3) Выполняется для бинарной диффузии или при наличии растворителя(вещество с высокой концентрацией)

 \vspace{20pt}

 В более общем случае для бинарной смеси вместо $n_{air}$ можно использовать $n_{summ}$, а вместо $v_{add}$  - $v_{relative}$
 
 \pagebreak

 \textit{Закон Фурье}

 Аналог закона Фика для теплопроводности: удельный поток тепла пропорционален градиенту температуры:
 \[ q_s = - \beta gradT\]

 $\beta$ - константа, зависящая от среды, в которой происходит теплообмен

 \vspace{30pt}
 
 Два сосуда равного объема соединены трубкой длины L и сечения S. Через некоторое время $\tau$ концентрации в сосудах должны выравняться

 Если объём трубки мал, то можно считать что концентрации в сосудах зависят только от времени, а выравнивание концентраций происходит за счёт диффузии в  трубке 

 \includegraphics[width=0.32\textwidth, right]{scheme} 

 Рассмотрим подзадачу диффузии в трубке

 Пусть концентрации на концах трубки - $n_1$ и $n_2$. Через время $\tau$ в трубке установится стационарный поток частиц, одинаковый в любом сечении трубки. Тогда из з-на Фика:
 
 \[ j = -D\frac{\partial{n}}{\partial{x}} = const\] 
 \[j = -D \frac{\Delta{n}}{L}\]

 Воспользуемся квазистационарным приближением. Будем считать, что в каждый момент времени в трубке успевает установиться стационарное течение. 
 К тому же, будем считать что концентрации в сосуде не зависит от x, то есть $n_1$ - концентрация в первом сосуде, $n_2$ - во втором. Тогда в первом сосуде число частиц $N_1 = n_1 * V$ , во втором $N_2 = n_2 * V$. За время $dt$ через трубку проходит $ j * S$ частиц. 
 \[\frac{dN}{dt} = jS\]

 Тогда изменение числа частиц в сосуде:

 \[\frac{d(\Delta{n})}{dt}  = 2jS = \frac{-2DS\Delta{n}}{VL} = \frac{\Delta{n}}{\tau}\]

 проинтегрируем

 \[\Delta{n} = \Delta{n_0}e^{-\frac{t}{\tau}}\]

 $\tau$ можно назвать характерным временем установления равновесия. $\tau = \frac{VL}{2DS}$

 \vspace{30pt}

 Проверим выполнение квазистатического приближения. Процесс должен быть достаточно продолжительным, то есть время установления равновесия должно быть много больше времени диффузии отдельной частицы: $\tau >> \tau_1$. По з-ну $\textit{Эйнштейна-Смолуховского}$ $\tau_1$ одного порядка с величиной $\frac{L^2}{2D}$.  Тогда $ V >> SL $, то есть объём трубки д.б. много меньше объёмов сосудов
 
 \pagebreak

 \textbf{Методика измерений}

 Датчики теплопроводности. Теплопроводность смеси зависит от концентраций. При малой концентрации примеси зависимость можно считать линейной(отклонение не более 0.5 процента):\par $\Delta{k} = const\Delta{n}$
\par
 В цилиндре, соединённом с сосудом, протянута проволока, которая нагревается током. Приращение температуры проволоки
 приводит к увеличению её сопротивления(которое в данных условиях меняется пропорционально теплопроводности газа). 
 Показания гальванометра в мостовой схеме зависят от состава газов, баланс(0 на гальванометре) достигается при равном составе, напряжение на гальванометре пропорционально разности теплопроводностей в сосудах, а значит и разности концентраций:
 \[U\propto \Delta{k} \propto \Delta{n}\]
 Таким образом напряжение должно меняться по тому же закону, что и $\Delta{n}$
 \[U = U_0e^{-\frac{t}{\tau}}\]
 Отсюда следует, что из зависимости U(t) можно оценить характерное время диффузии $\tau$

\vspace{10pt}

\textbf{Установка}

\includegraphics[width=0.6\textwidth]{scheme1} 
\includegraphics[width=0.45\textwidth]{most} 

\pagebreak

Данные:

Объём сосудов - $(775\pm10)$ $cm^3$ \par 
Длина трубки к сечению трубки - $(5,3\pm0,1)$ $\frac{1}{cm}$\par
\vspace{6pt}

Манометр: ц.д. - 7.5 тор, \par
Вольтметр: цифровой(4 знака после запятой) , \par
Давления(в торрах) - 40, 50, 60, 70, 100 \par

\vspace{30pt}

Проверить выполнение экспоненциальной зависимости напряжения от времени(то есть и концентраций)\par
Проверить истинность начальной теории, предполагающей в опыте диффузию примеси лёгких частиц на фоне неподвижных частиц воздуха,
которая предсказывает обратную пропорциональность коэффицента диффузии и давления

Рассчётные значения:

\begin{center}
    \begin{tabular}{ | m{5em} | m{3cm} | m{3cm} | }
        \hline
        P, torr & $\tau$, c & D, $\frac{m^2}{c} * 10^-3$ \\
        \hline
        40 & 211 & 1.946 \\
        \hline
        50 & 209 & 1.965\\
        \hline
        60 & 271 & 1.515\\ 
        \hline
        70 & 309 & 1.329 \\
        \hline
        100 & 420 & 0.978\\
        \hline
    \end{tabular}
\end{center}

\pagebreak

Графики ln(U) от t \par
\includegraphics[width=1.2\textwidth]{40tor.png}\par
\includegraphics[width=1.2\textwidth]{70torr.png}

Графики показывают экспоненциальную зависимость U от t. Можно оценить $\tau$ как $\frac{\Delta{t}}{\Delta{ln(U)}}$

\pagebreak

\includegraphics[width=1.2\textwidth]{t(p)}
\includegraphics[width=1.2\textwidth]{d(1p)}

\pagebreak

Оценим погрешность $\tau$. Учитывая ожидаемую линейную зависимость U(t), точность оценки $\tau$ зависит от полученных эксперементальных точек.
Эту погрешность оценим с помощью метода наименьших квадратов(учитывая, что погрешность измерений вольтметра пренебрежимо мала) \par

\[\sigma^2{\tau} = \sigma^2{U} + \sigma^2(\frac{1}{k})\]

Оценим погрешность $D = \frac{VL}{2\tau{s}}$ \par
$\sigma^2{D} = \sigma^2{V} + \sigma^2{\frac{L}{S}} + \sigma^2{\tau}$ 

Итоговые значения:

\begin{center}
    \begin{tabular}{ | m{3cm} | m{4cm} | m{4cm} | }
        \hline
        P, tor & $\tau$, c & D, $\frac{m^2}{c}$ $* 10^-5$ \\
        \hline
        40 & $211\pm$ 12 & $1.946 \pm 0.121$\\
        \hline
        50 & 209$\pm$ 17 & $1.964 \pm 0.260$ \\
        \hline
        60 & 271$\pm$ 12 & $1.515 \pm 0.075$\\ 
        \hline
        70 & 309$\pm$ 9 & $1.329 \pm 0.045$\\
        \hline
        100 & 420$\pm$ 12 & $0.978 \pm 0.039$\\
        \hline
    \end{tabular}
\end{center}

Экстраполируя данные, полученные из графиков$(D(\frac{1}{p}), \tau(p))$, получим 
коэффицент диффузии при нормальном атмосферном давлении 746 тор.

\[\tau_{746} = 2760c, \hspace{2pt} D_{746} = (7,44 \pm 0,16) * 10^{-5} \frac{m^2}{c}\]

Табличное значение коэффицента диффузии: $6,2 * 10^{-5} \frac{m^2}{c}$

\vspace{20pt}

Таким образом мы проверили выполнение закона Фика для диффузии бинарной смеси(получена экспоненциальная зависимость концентрации от времени),
а также, за исключением 1 точки(p = 40 тор), убедились в том что для диффузии в растворителе верно соотношение 
$D = \frac{1}{3}\vartheta\lambda$ (обратная пропорциональность D и P). С помощью полученных данных удалось экстраполировать
значение коэффицента диффузии с относительной точностью $19\%$

\end{document}